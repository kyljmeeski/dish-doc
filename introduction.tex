\section*{\centering Введение}
\addcontentsline{toc}{section}{Введение}

Представьте себе такую ситуацию: вам звонят с незнакомого номера и уверенно сообщают, что ваш счет заблокирован. Чтобы восставновить доступ, нужно выполнить несколько действий, которые кажутся совершенно безобидными. В панике, не задумываясь о последствиях, вы следуете указаниям собеседника. Спустя несколько минут вы обнаруживаете нули на своих счетах и кошельках. И таких, как вы, тысячи по всей стране, а общая сумма украденных средств исчисляется миллионами долларов. Это развязка фильма режиссёра Дэвида Эйера «Пчеловод» \cite{beekeeper2024}, где главный герой восстанавливает справедливость, разрушая целую мошенническую империю. Однако в реальной жизни герой \cite{fraud2024}, готовый навести порядок, появляется не всегда, а вот мошенники – всегда рядом.

https://aag-it.com/the-latest-phishing-statistics
За один год было раскрыто почти 1 миллиард электронных писем, что затронуло каждого пятого пользователя Интернета.

https://aag-it.com/the-latest-cyber-crime-statistics
В 2021 году учетные записи каждого второго американского интернет-пользователя были взломаны.
В 2023 году 50\% британских предприятий в той или иной форме подверглись кибератакам.

https://www.statista.com/statistics/267132/total-damage-caused-by-by-cybercrime-in-the-us
В 2023 году денежный ущерб от киберпреступлений, о которых сообщается в американском Центре жалоб на интернет-преступления (IC3) составил 10,3 миллиарда долларов США.

https://csis-website-prod.s3.amazonaws.com/s3fs-public/publication/economic-impact-cybercrime.pdf
Фишинг остается самым популярным и простым способом совершения киберпреступлений. Ежедневно фиксируется 33 000 случаев фишинговых атак.

https://www.statista.com/statistics/1306269/volume-vishing-attacks-organizations
Согласно опросам взрослых и ИТ-специалистов, проведенным в 2023 году, почти семь из десяти респондентов сообщили, что сталкивались с вишинговыми атаками.

https://24.kg/obschestvo/256845_okolo_100_tyisyach_kyirgyizstantsev_stali_jertvami_kiberatak_v2021_godu
Около 100 тысяч кыргызстанцев стали жертвами кибератак в 2021 году.

https://mir24.tv/news/16597425/postradavshih-ot-telefonnyh-moshennikov-v-kyrgyzstane-stalo-v-dva-raza-bolshe
За полгода в стране с заявлением об онлайн-мошенничестве обратились больше 800 человек, это вдвое больше, чем за аналогичный период прошлого года. Общая сумма ущерба – около 45 млн сомов, это примерно столько же рублей.

Телефонное мошенничество – это форма фишинговых атак\footnote{Фишинг –  вид интернет-мошенничества, целью которого является получение доступа к конфиденциальным данным пользователей – логинам и паролям.}, которая осуществляются по телефону, с целью получения конфиденциальной информации жертвы. Мошенники используют эту информацию для личной выгоды, включая материальное обогащение.

Количество жертв телефонного мошенничества в Кыргызстане за последний год увеличилось вдвое. Так с начала 2024 года Министерство внутренних дел страны возбудило более 800 уголовных дел по фактам телефонного мошенничества. В общей сложности жители потерпели ущерб в размере около 45 млн сомов.

Во введении приводится актуальность темы исследования, цель и задачи ВКР, объект и предмет исследования, структура ВКР.

Актуальность темы исследования:
Для раскрытия актуальности темы исследования надо ответить на следующие вопросы:
\begin{itemize}
	\item Почему это является проблемой?
	\item Как часто это происходит?
	\item Кто наиболее подвержен этому?
	\item Какие последствия?
\end{itemize}

\newpage