\section*{\centering Введение}
\addcontentsline{toc}{section}{Введение}

Представьте себе такую ситуацию: вам звонят с незнакомого номера и уверенно сообщают, что ваш счет заблокирован. Чтобы восставновить доступ, нужно выполнить несколько действий, которые кажутся совершенно безобидными. В панике, не задумываясь о последствиях, вы следуете указаниям собеседника. Спустя несколько минут вы обнаруживаете нули на своих счетах и кошельках. И таких, как вы, тысячи по всей стране, а общая сумма украденных средств исчисляется миллионами долларов. Это развязка фильма режиссёра Дэвида Эйера «Пчеловод» \cite{beekeeper2024}, где главный герой восстанавливает справедливость, разрушая целую мошенническую империю. Однако в реальной жизни герой \cite{fraud2024}, готовый навести порядок, появляется не всегда, а вот мошенники – всегда рядом.

Согласно последним отчетам, в 2024 году было вскрыто письмо каждого пятого пользователя Интернета, что составляет почти 1 миллиард электронных писем в год \cite{griffiths2024phishing}. Кроме того, каждый второй американец стал жертвой взлома аккаунта, в то же время половина британских предприятий столкнулась с кибератаками в той или иной форме \cite{griffiths2024cybercrime}. В 2023 году сумма ущерба от киберпреступлений, зарегистрированных в американском Центре жалоб на интернет-преступления (IC3), составила 10,3 миллиарда долларов США \cite{petrosyan2023}. Стоит отметить, что фишинг\footnote{Фишинг –  вид интернет-мошенничества, целью которого является получение доступа к конфиденциальным данным пользователей – логинам и паролям.} является наиболее распространённым и доступным методом киберпреступлений, с ежедневной регистрацией 33 000 случаев фишинговых атак \cite{lewis2018}. 

В 2021 году около 100 тысяч граждан Кыргызстана стали жертвами кибератак \cite{kopytin2023}. А за последний год количество жертв телефонного мошенничества увеличилось вдвое. Так с начала 2024 года Министерство внутренних дел возбудило более 800 уголовных дел по фактам телефонного мошенничества. В общей сложности жители потерпели ущерб в размере около 45 миллионов сомов \cite{amatbekova2024}. По данным опросов среди взрослых и ИТ-специалистов, проведенных в 2023 году, почти семь из десяти респондентов признались, что сталкивались с вишинговыми атаками \cite{petrosyan2024}.

Вишинг или телефонное мошенничество – это форма фишинговых атак, которая осуществляются с использованием голоса – по телефону или через голосовые сообщения – с целью получения конфиденциальной информации жертвы. Мошенники используют эту информацию для личной выгоды, включая материальное обогащение.

Цель – разработать система, которая будет защищать людей от телефонных мошенников.

Задачи:
- Определить, является ли звонок мошенническим
- Составить реестр номеров мошенников


Во введении приводится актуальность темы исследования, цель и задачи ВКР, объект и предмет исследования, структура ВКР.

\newpage