\section*{\centering Введение}
\addcontentsline{toc}{section}{Введение}

Представьте себе такую ситуацию: вам звонят с незнакомого номера и уверенно сообщают, что ваш счет заблокирован, просят выполнить несколько действий, которые кажутся совершенно безобидными, чтобы восстановить доступ. В панике, не задумываясь о последствиях, вы следуете указаниям собеседника. Спустя несколько минут вы обнаруживаете нули на всех своих счетах и кошельках. И таких, как вы, тысячи по всей стране, а общая сумма украденных средств исчисляется миллионами долларов. Это развязка фильма режиссёра Дэвида Эйера «Пчеловод» \cite{beekeeper2024}, где главный герой восстанавливает справедливость, разрушая целую мошенническую империю. Однако в реальной жизни герой \cite{fraud2024}, готовый навести порядок, появляется не всегда, а вот мошенники – всегда рядом.

Телефонное мошенничество – это форма фишинговых атак\footnote{Фишинг –  вид интернет-мошенничества, целью которого является получение доступа к конфиденциальным данным пользователей – логинам и паролям.}, которая осуществляются по телефону, с целью получения конфиденциальной информации жертвы. Мошенники используют эту информацию для личной выгоды, включая материальное обогащение.

Количество жертв телефонного мошенничества в Кыргызстане за последний год увеличилось вдвое. Так с начала 2024 года Министерство внутренних дел страны возбудило более 800 уголовных дел по фактам телефонного мошенничества. В общей сложности жители потерпели ущерб в размере около 45 млн сомов.

Во введении приводится актуальность темы исследования, цель и задачи ВКР, объект и предмет исследования, структура ВКР.

Актуальность темы исследования:
Для раскрытия актуальности темы исследования надо ответить на следующие вопросы:
\begin{itemize}
	\item Почему это является проблемой?
	\item Как часто это происходит?
	\item Кто наиболее подвержен этому?
	\item Какие последствия?
\end{itemize}

\newpage