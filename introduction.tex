\section*{\centering Введение}
\addcontentsline{toc}{section}{Введение}

Представьте себе такую ситуацию: вам звонят с незнакомого номера и уверенно сообщают, что ваш счет заблокирован. Чтобы восставновить доступ, нужно выполнить несколько действий, которые кажутся совершенно безобидными. В панике, не задумываясь о последствиях, вы следуете указаниям собеседника. Спустя несколько минут вы обнаруживаете нули на своих счетах и кошельках. И таких, как вы, тысячи по всей стране, а общая сумма украденных средств исчисляется миллионами долларов. Это развязка фильма режиссёра Дэвида Эйера «Пчеловод» \cite{beekeeper2024}, где главный герой восстанавливает справедливость, разрушая целую мошенническую империю. Однако в реальной жизни герой \cite{fraud2024}, готовый навести порядок, появляется не всегда, а вот мошенники – всегда рядом.

Согласно последним отчетам, в 2024 году было вскрыто письмо каждого пятого пользователя Интернета, что составляет почти 1 миллиард электронных писем [https://aag-it.com/the-latest-phishing-statistics]. 

Каждый второй американец стал жертвой взлома аккаунта, в то же время половина британских предприятий столкнулась с кибератаками в той или иной форме [https://aag-it.com/the-latest-cyber-crime-statistics]. 

В 2023 году сумма ущерба от киберпреступлений, зарегистрированных в американском Центре жалоб на интернет-преступления (IC3), составила 10,3 миллиарда долларов США [https://www.statista.com/statistics/267132/total-damage-caused-by-by-cybercrime-in-the-us].

Фишинг продолжает быть наиболее распространённым и доступным методом киберпреступлений, с ежедневной регистрацией 33 000 случаев фишинговых атак [https://csis-website-prod.s3.amazonaws.com/s3fs-public/publication/economic-impact-cybercrime.pdf].

По данным опросов среди взрослых и ИТ-специалистов, проведенных в 2023 году, почти семь из десяти респондентов признались, что сталкивались с вишинговыми атаками [https://www.statista.com/statistics/1306269/volume-vishing-attacks-organizations].

В 2021 году около 100 тысяч граждан Кыргызстана стали жертвами кибератак [https://24.kg/obschestvo/256845_okolo_100_tyisyach_kyirgyizstantsev_stali_jertvami_kiberatak_v2021_godu].

Количество жертв телефонного мошенничества в Кыргызстане за последний год увеличилось вдвое. Так с начала 2024 года Министерство внутренних дел страны возбудило более 800 уголовных дел по фактам телефонного мошенничества. В общей сложности жители потерпели ущерб в размере около 45 млн сомов [https://mir24.tv/news/16597425/postradavshih-ot-telefonnyh-moshennikov-v-kyrgyzstane-stalo-v-dva-raza-bolshe].

Телефонное мошенничество – это форма фишинговых атак\footnote{Фишинг –  вид интернет-мошенничества, целью которого является получение доступа к конфиденциальным данным пользователей – логинам и паролям.}, которая осуществляются по телефону, с целью получения конфиденциальной информации жертвы. Мошенники используют эту информацию для личной выгоды, включая материальное обогащение.



Во введении приводится актуальность темы исследования, цель и задачи ВКР, объект и предмет исследования, структура ВКР.

Актуальность темы исследования:
Для раскрытия актуальности темы исследования надо ответить на следующие вопросы:
\begin{itemize}
	\item Почему это является проблемой?
	\item Как часто это происходит?
	\item Кто наиболее подвержен этому?
	\item Какие последствия?
\end{itemize}

\newpage